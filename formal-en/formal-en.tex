\documentclass{resume}
\usepackage{setspace}
\renewcommand{\baselinestretch}{1.0}
\begin{document}

\name{\textbf{\href{https://fassial.github.io}{Hui Zheng}}}
    
\contactinfo{Phone: +86 137-9231-5475}{Email: \href{mailto:fassial19991217@gmail.com}{fassial19991217@gmail.com}}{}{}

\section{{\bfseries Research Interests}}
\begin{itemize}[parsep=0.2ex]
\item \textbf{Neuroscience}: Computational Neuroscience; Brain-Inspired Computing
\end{itemize}

\section{{\bfseries Education}}
\datedsubsection{{\bfseries Wuhan University}}{}{Wuhan, China}
\datedsubsection{}{B.S in Computer Science and Technology,Hongyi Honor College}{Sept. 2017 - Jun. 2021(expected)}
\begin{itemize}[parsep=0.1ex]
    \item \textbf{GPA}: 3.84/4.00(92.1/100)
    \item \textbf{Rank}: 2/32(selected from 363 students in School of Computer Science, Wuhan University)
    \item \textbf{Exchange}: Visiting Student at University of California, Berkeley(2019 summer)
\end{itemize}

\section{{\bfseries Publications}}
\begin{itemize}[parsep=0.2ex]
    \item Yunzhe Li*, \textbf{Hui Zheng*}, He Zhu*, Haojun Ai and Xiaowei Dong. "Cross-People Mobile-Phone Based Airwriting Character Recognition". ICPR2020 Accepted.
    \item Wenquan Xu, Haoyu Song, Linyang Hou, \textbf{Hui Zheng}, Xinggong Zhang, Chuwen Zhang, Wei Hu, Yi Wang, Bin Liu. "SODA: Similar 3D Object Detection Accelerator at Network Edge for Autonomous Driving". INFOCOM2021 Accepted.
\end{itemize}

%% research experience
\section{{\bfseries Research Experience}}
% pku-si wu
\datedsubsection{\textbf{A Neural Network Model with Gap Junction for Global Feature Extraction}, Research Intern}{}{}
\datedsubsection{}{Academy for Advanced Interdisciplinary Studies, Peking University}{}
\datedsubsection{}{Advised by \href{http://www.aais.pku.edu.cn/duiwu/showproduct.php?id=208}{Prof. Si Wu}}{Mar. 2021 - now}
% nibs-minmin luo
\datedsubsection{\textbf{Single-Cell Transcriptomics to uncover the Relationships between Inflammation and Hormone in Pituitary Cells}, Research Intern}{}{}
\datedsubsection{}{National Institute of Biological Sciences, Beijing}{}
\datedsubsection{}{Advised by \href{http://www.nibs.ac.cn/yjsjyimgshow.php?cid=5&sid=6&id=775}{Prof. Minmin Luo}}{Sept. 2020 - Mar. 2021}
\begin{itemize}[parsep=0.2ex]
    \item Research on issues related to the role of pituitary cells in systemic neuroinflammation at the single-cell transcriptome level.
    \item We revealed the transcriptional differences of different types of pituitary cells in the process of central nervous endocrine inflammation regulation. And we discovered a group of transcription factors uniformly expressed in different types of pituitary cells.
\end{itemize}
% thu-bin liu
\datedsubsection{\textbf{SODA: Similar 3D Object Detection Accelerator at Network Edge for Autonomous Driving}, Research Intern}{}{}
\datedsubsection{}{School of Computer Science, Tsinghua University}{}
\datedsubsection{}{Advised by \href{http://www.cs.tsinghua.edu.cn/publish/cs/4616/2013/20130424093153561198286/20130424093153561198286_.html}{Prof. Bin Liu}}{May. 2020 - Aug. 2020}
\begin{itemize}[parsep=0.2ex]
    \item Research on issues related to the real-time processing of autonomous driving in the Internet of Vehicles.
    \item SODA accelerates the MEC-assisted similar 3D object detection for autonomous driving. We designed efficient algorithms for the novel TCAM-NMC in-network accelerator, and through extensive evaluations, confirmed the architecture feasibility and performance superiority on the subject matter.
\end{itemize}
% whu-haojun ai
\datedsubsection{\textbf{Cross-People Mobile-Phone Based Airwriting Character Recognition}, Research Intern}{}{}
\datedsubsection{}{School of Cyber Science and Engineering, Wuhan University}{}
\datedsubsection{}{Advised by \href{http://cse.whu.edu.cn/index.php?s=/home/szdw/detail/id/95.html}{A/Prof. Haojun Ai}}{Feb. 2020 - Apr. 2020}
\begin{itemize}[parsep=0.2ex]
    \item Research on issues related to transfer learning in Air-Writing.
    \item We developed a system that could transfer between different people. This system has better personalized recognition performance.
\end{itemize}
% whu-shubo liu
\datedsubsection{\textbf{RISC-V Super Scalar Processor Design and Internet of Things Application}, Research Intern}{}{}
\datedsubsection{}{School of Computer Science, Wuhan University}{}
\datedsubsection{}{Co-advised by \href{http://cs.whu.edu.cn/teacherinfo.aspx?id=309}{Prof. Shubo Liu} and \href{http://cs.whu.edu.cn/teacherinfo.aspx?id=266}{A/Prof. Zhaohui Cai}}{May. 2019 - Jan. 2020}

%% projects
\section{{\bfseries Projects}}
% whu-airwriting
\datedsubsection{\href{https://github.com/Fassial/NUFIC2019-WHU}{\textbf{Air-Writing Recognition based on Deep Learning}}, Team Leader}{}{}
\datedsubsection{}{School of Cyber Science and Engineering, Wuhan University}{}
\datedsubsection{}{Advised by \href{http://cse.whu.edu.cn/index.php?s=/home/szdw/detail/id/95.html}{A/Prof. Haojun Ai}}{Oct. 2019 - Dec. 2019}
\begin{itemize}[parsep=0.2ex]
    \item Works of \href{http://fpga.icisc.cn/}{FPGA Innovation Design Competition}. Use Bluetooth ring for Air-Writing. A more natural way of Human-Computer Interaction.
    \item Use FPGA to filter and unpack the data collected by the acceleration sensor. Then transfer the data to the embedded Arm and use the deep neural network for prediction.
\end{itemize}
% whu-mips
\datedsubsection{\href{https://github.com/trifling-mips}{\textbf{2-issue MIPS-CPU}}, Team Leader}{}{}
\datedsubsection{}{School of Computer Science, Wuhan University}{}
\datedsubsection{}{Co-advised by \href{http://cs.whu.edu.cn/teacherinfo.aspx?id=309}{Prof. Shubo Liu} and \href{http://cs.whu.edu.cn/teacherinfo.aspx?id=266}{A/Prof. Zhaohui Cai}}{Jan. 2020 - Aug. 2020}
\begin{itemize}[parsep=0.2ex]
    \item Works of \href{http://www.nscscc.org/}{NSCSCC2020}. Use MIPS32 ISA. Run at 80MHz.
    \item Support all instructions necessary to start the linux kernel. Parallelize TLB and cache. Reach 20 points in the NSCSCC performance test.
\end{itemize}
% whu-lcore
\datedsubsection{\href{https://github.com/Fassial/Lcore}{\textbf{Lcore}}, Team Leader}{}{}
\datedsubsection{}{School of Computer Science, Wuhan University}{}
\datedsubsection{}{Co-advised by \href{http://cs.whu.edu.cn/teacherinfo.aspx?id=309}{Prof. Shubo Liu} and \href{http://cs.whu.edu.cn/teacherinfo.aspx?id=266}{A/Prof. Zhaohui Cai}}{Aug. 2019 - Sept. 2019}
\begin{itemize}[parsep=0.2ex]
    \item Works of \href{http://www.nscscc.org/}{NSCSCC2019}. A simple operating system, running on MIPS-CPU.
    \item Support basic process switching, memory management and shell interaction, etc.
\end{itemize}
% whu-oddb
\datedsubsection{\href{https://github.com/Fassial/ODDB-Lab}{\textbf{Object-Deputy DataBase}}, Team Member}{}{}
\datedsubsection{}{School of Computer Science, Wuhan University}{}
\datedsubsection{}{Advised by \href{http://cs.whu.edu.cn/teacherinfo.aspx?id=297}{Prof. Zhiyong Peng}}{Jan. 2020 - Mar. 2020}
\begin{itemize}[parsep=0.2ex]
    \item The design of Database Design and Implementation Course.
    \item Realize basic operations of ODDB, such as adding, deleting, modifying and searching.
\end{itemize}

%% awards & scholarships
\section{{\bfseries Awards \& Scholarships}}
% 2021
\datedsubsection{Outstanding Graduate \textbf{(12 out of 127, 10\%)}, Wuhan University}{}{Apr. 2021}
% 2020
\datedsubsection{National Scholarship, Wuhan University}{}{Oct. 2020}
\datedsubsection{Excellent Student Scholarship \textbf{(Rank: 1/32)}, Wuhan University}{}{Oct. 2020}
% 2019
\datedsubsection{National Second Prize of FPGA Innovation Design Competition, China}{}{Dec. 2019}
\datedsubsection{National Second Prize of Intelligent Robot Fighting Competition, China}{}{Oct. 2019}
\datedsubsection{Excellent Student Scholarship \textbf{(Rank: 4/32)}, Wuhan University}{}{Oct. 2019}
% 2018
\datedsubsection{Excellent Student Scholarship \textbf{(Rank: 8/32)}, Wuhan University}{}{Oct. 2018}
% 2017
\datedsubsection{Freshman Scholarship, Wuhan University}{}{Oct. 2017}

%% skills
\section{{\bfseries Skills}}
\begin{itemize}[parsep=0.2ex]
    \item \textbf{Programming}: systemVerilog, C, python, R, java, LaTex, javascript, matlab
    \item \textbf{Development Framework}: pytorch, tensorflow, vue
    \item \textbf{English Level}: \textbf{CET-4} (538), \textbf{CET-6} (533)
\end{itemize}

%% leadership
\section{{\bfseries Leadership}}
% hongyi honor college
\datedsubsection{\textbf{Minister of Network Technology Department of Student Union}}{}{}
\datedsubsection{}{Hongyi Honor College, Wuhan University}{Sept. 2018 - Jun. 2019}
% WHU-MSC
\datedsubsection{\textbf{Vice-Chairman of Microsoft Student Club}}{}{}
\datedsubsection{}{Wuhan University}{Sept. 2019 - Jun. 2020}

\end{document}
